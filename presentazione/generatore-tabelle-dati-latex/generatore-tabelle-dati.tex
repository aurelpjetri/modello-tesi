\documentclass[a4paper, 12pt]{article}
\usepackage{graphicx}
\usepackage[italian]{babel}
\usepackage[T1]{fontenc}
\usepackage[utf8]{inputenc}

\author{Aurel Pjetri}
\date{}

\begin{document}

\begin{table}[h]
  \centering
  \resizebox{0.7\textwidth}{!} {
  \begin{tabular}{ |l|l|l| }
	\hline
	\textbf{Scenario}	&		\textbf{Densità costante}	&		\textbf{Transitorio}	\\ \hline
	\textbf{Monolitico} &		 						&		9949		\\ \hline
	\textbf{A}			 &		395650 				&		15998		\\ \hline
	\textbf{D}			 &		408295 				&		15862		\\ \hline

  \end{tabular}
  }
  \caption{Confronto numero totale di dati raccolti dalle simulazioni}
  \label{tab:tabella-confronto}
\end{table}

\begin{table}[h]
  \centering
  \resizebox{0.7\textwidth}{!} {
  \begin{tabular}[t]{ |l|l|l| }
	\hline
	\textbf{Regione}	 &		\textbf{Densità costante}	&		\textbf{Transitorio}	\\ \hline
	\textbf{A1} 		 &		51325				&		1693		\\ \hline
	\textbf{A2}			 &		55661 				&		1528		\\ \hline
	\textbf{A3}			 &		69095 				&		2423		\\ \hline
	\textbf{A4}			 &		71533 				&		4057		\\ \hline
	\textbf{A5}			 &		62510 				&		2582		\\ \hline
	\textbf{A6}			 &		85526 				&		3715		\\ \hline

  \end{tabular}
  
  }
  \caption{Numero di dati raccolti per lo scenario A nei due modelli}
  \label{tab:tabella-dati-simulazioni-A}
\end{table}

\begin{table}[h]
  \centering
  \resizebox{0.7\textwidth}{!} {
 \begin{tabular}[t]{ |l|l|l| }
	 	\hline
	\textbf{Regione}	&		\textbf{Densità costante}	&		\textbf{Transitorio}	\\ \hline
	\textbf{D1} 	 &		23463					&	803	\\ \hline
	\textbf{D2}		 &		27834 				&		803		\\ \hline
	\textbf{D3}		 &		46190 				&		803		\\ \hline
	\textbf{D4}		 &		35894 				&		803		\\ \hline
	\textbf{D5}		 &		46454 				&		1100		\\ \hline
	\textbf{D6}		 &		43967 				&		1100		\\ \hline
	\textbf{D7}		 &		25052 				&		1900		\\ \hline
	\textbf{D8}		 &		25902 				&		1900		\\ \hline
	\textbf{D9}		 &		35990 				&		1450		\\ \hline
	\textbf{D10}	 &		32826 				&		1450		\\ \hline
	\textbf{D11}	 &		43770 				&		1450		\\ \hline
	\textbf{D12}	 &		20953 				&		2300		\\ \hline
	
	
  \end{tabular}
  
  }
  \caption{Numero di dati raccolti per lo scenario D nei due modelli}
  \label{tab:tabella-dati-simulazioni-D}
\end{table}

\begin{table}[h]
  \centering
  \resizebox{0.7\textwidth}{!} {
 \begin{tabular}[t]{ |l|l|l| }
	 	\hline
	\textbf{Regione}	&		\textbf{Densità costante}	&		\textbf{Transitorio}	\\ \hline
	\textbf{D1} 	 &		0:27:26					&		0:01:56		\\ \hline
	\textbf{D2}		 &		0:29:15 				&		0:02:03		\\ \hline
	\textbf{D3}		 &		0:49:46 				&		0:01:58		\\ \hline
	\textbf{D4}		 &		0:29:48 				&		0:02:00		\\ \hline
	\textbf{D5}		 &		1:12:14 				&		0:01:51		\\ \hline
	\textbf{D6}		 &		1:04:18 				&		0:01:40		\\ \hline
	\textbf{D7}		 &		1:14:06 				&		0:01:58		\\ \hline
	\textbf{D8}		 &		1:04:41 				&		0:02:07		\\ \hline
	\textbf{D9}		 &		1:08:36 				&		0:01:58		\\ \hline
	\textbf{D10}	 &		0:53:40 				&		0:02:06		\\ \hline
	\textbf{D11}	 &		0:06:25 				&		0:01:43		\\ \hline
	\textbf{D12}	 &		0:08:47 				&		0:01:48		\\ \hline
	
	
  \end{tabular}
  
  }
  \caption{Tempi di simulazione relativi alle singole regioni dello scenario D con metodo a densità costante e transitorio}
  \label{tab:tabella-dati-simulazioni-D}
\end{table}

\begin{table}[h]
  \centering
  \resizebox{0.7\textwidth}{!} {
  \begin{tabular}{ |l|l|l| }
	\hline
	\textbf{Scenario}	&		\textbf{Densità costante}	&		\textbf{Transitorio}	\\ \hline
	\textbf{Monolitico} &		 						&		9:19:40		\\ \hline
	\textbf{A}			 &		21:33:36 				&		0:21:51		\\ \hline
	\textbf{D}			 &		09:09:02 				&		0:23:13		\\ \hline

  \end{tabular}
  }
  \caption{Confronto dei tempi complessivi di simulazione}
  \label{tab:tabella-confronto}
\end{table}


\end{document}