\documentclass[a4paper, 12pt]{article}
\usepackage{graphicx}
\usepackage[italian]{babel}
\usepackage[T1]{fontenc}
\usepackage[utf8]{inputenc}

\author{Aurel Pjetri}
\date{}

\begin{document}

\begin{table}[h]
  \centering
  \resizebox{0.7\textwidth}{!} {
  \begin{tabular}{ |l|l|l| }
	\hline
	\textbf{Scenario}	&		\textbf{densità costante}	&		\textbf{transitorio}	\\ \hline
	\textbf{Monolitico} &		 						&		9949		\\ \hline
	\textbf{A}			 &		395650 				&		15998		\\ \hline
	\textbf{D}			 &		408295 				&		9636		\\ \hline

  \end{tabular}
  }
  \caption{Confronto numero totale di dati raccolti dalle simulazioni}
  \label{tab:tabella-confronto}
\end{table}

\begin{table}[h]
  \centering
  \resizebox{0.7\textwidth}{!} {
  \begin{tabular}[t]{ |l|l|l| }
	\hline
	\textbf{Regione}	 &		\textbf{Densità costante}	&		\textbf{Transitorio}	\\ \hline
	\textbf{A1} 		 &		51325				&		1693		\\ \hline
	\textbf{A2}			 &		55661 				&		1528		\\ \hline
	\textbf{A3}			 &		69095 				&		2423		\\ \hline
	\textbf{A4}			 &		71533 				&		4057		\\ \hline
	\textbf{A5}			 &		62510 				&		2582		\\ \hline
	\textbf{A6}			 &		85526 				&		3715		\\ \hline

  \end{tabular}
  
  }
  \caption{Numero di dati raccolti per lo scenario A nei due modelli}
  \label{tab:tabella-dati-simulazioni-A}
\end{table}

\begin{table}[h]
  \centering
  \resizebox{0.7\textwidth}{!} {
 \begin{tabular}[t]{ |l|l|l| }
	 	\hline
	\textbf{Regione}	&		\textbf{Densità costante}	&		\textbf{Transitorio}	\\ \hline
	\textbf{D1} 	 &		23463					&	803	\\ \hline
	\textbf{D2}		 &		27834 				&		803		\\ \hline
	\textbf{D3}		 &		46190 				&		803		\\ \hline
	\textbf{D4}		 &		35894 				&		803		\\ \hline
	\textbf{D5}		 &		46454 				&		803		\\ \hline
	\textbf{D6}		 &		43967 				&		803		\\ \hline
	\textbf{D7}		 &		25052 				&		803		\\ \hline
	\textbf{D8}		 &		25902 				&		803		\\ \hline
	\textbf{D9}		 &		35990 				&		803		\\ \hline
	\textbf{D10}	 &		32826 				&		803		\\ \hline
	\textbf{D11}	 &		43770 				&		803		\\ \hline
	\textbf{D12}	 &		20953 					&		803		\\ \hline
	
	
  \end{tabular}
  
  }
  \caption{Numero di dati raccolti per lo scenario D nei due modelli}
  \label{tab:tabella-dati-simulazioni-D}
\end{table}

\end{document}